%==============================================================================
% MustHave strings in file
%==============================================================================

\documentclass[12pt, a4paper, oneside]{report}	% {article|letter|report}
\usepackage[utf8]{inputenc}
\usepackage[T1, T2A]{fontenc}
\usepackage[english, russian]{babel}


%==============================================================================
% Page properties
%==============================================================================
\usepackage{geometry}
\geometry
{
	a4paper,
	left=20mm,
	right=20mm,
	top=25mm,
	bottom=20mm,
	headheight=10pt,
	footskip=20pt
}

 % or

\textheight=28cm			% высота текста A4
\textwidth=16cm				% ширина текста A4
\oddsidemargin=0pt			% отступ от левого края
\topmargin=0cm			% отступ от верхнего края



\parindent=0pt				% абзацный отступ
\parskip=0pt				% интервал между абзацами
\tolerance=2000				% терпимость к "жидким" строкам
\flushbottom				% выравнивание высоты страниц
\setcounter{secnumdepth}{0}	% секции без нумерации


%==============================================================================
% Footer and header
%==============================================================================

\usepackage{fancybox, fancyhdr}

% For first page
\fancypagestyle{firststyle} %новый стиль
{
	\fancyfootoffset[R]{-12cm} %так можно регулировать ширину колонтитула
	% ...
	\renewcommand{\footrulewidth}{0.3 mm} %толщина отделяющей полоски снизу
	\renewcommand{\headrulewidth}{0.3 mm} %толщина отделяющей полоски сверху
}

\begin{document}
\thispagestyle{firststyle}

% Example:
\pagestyle{fancy}
\fancyhead[L]{{\small Подготовлено для Игната, 08.04.20}}
\fancyhead[R]{{\small denisov0gleb@gmail.com}}

%\fancyhead{} % Clear the header

%E: Even page (чётная страница)
%O: Odd page (нечётная страница)
%L: Left field (левое поле)
%C: Center field (центральное поле)
%R: Right field  (правое поле)
%H: Header  (верхний колонтитул)
%F: Footer   (нижний колонтитул)
\fancyhead[*]{}
\fancyfoot[*]{}

\fancyfoot[C]{-~\thepage~-}

% Delete the line of header/footer
\renewcommand{\headrulewidth}{0pt}


%==============================================================================
% Styles
%==============================================================================
% italic
\textit{italic}

% special
\emph{special}

% Capital
\textsc{capital}

% bold
\textbf{bold}

% cross out words
\usepackage{ulem}
\sout{Wrong.} Correct.

% colored text
\usepackage{xcolor}
\textcolor{red}{red text}

% Correction on 
\marginpar{Этот текст появится сбоку на полях.}

{\large text}

%or

\begin{large}
	some text
\end{large}

% text in color box
\newcommand\marker[2]{{\fboxsep=0pt\colorbox{#1}{\strut #2}}}
\marker[red]{text}

%or simply but with additional spaces

\colorbox{red!50}{ugly text}
%==============================================================================
% Math
%==============================================================================

\usepackage{amsmath}
\usepackage{amsfonts}
\usepackage{amssymb}
\usepackage[warn]{mathtext}

% In-text equation:
$ ... $

% Centered equation:
$$ ... $$

% Equations
\begin{equation}
	%...
\end{equation}

%Russian text in the equation
\text{}

\begin{eqnarray}
	%long string of equation
	%or many lines
\end{eqnarray}

%Systems
$\begin{cases}
	4x + 3y = 6\\
	6x - 5y + 29= 0
\end{cases}$


%==============================================================================
% Symbols
%==============================================================================

\usepackage{amsfonts}
\usepackage{amssymb}

%Detect the hand written symbol
%http://detexify.kirelabs.org/classify.html

% Grad symbol:
^\circ

%MxN:
\times

%Brackets
(\big (\Big (\bigg (\Bigg

% № in russian text:
\usepackage{textcomp}

\newcommand*{\No}{\textnumero}

\No

% /
{\slash}

% ...
\ldots


%==============================================================================
% References
%==============================================================================
\usepackage{hyperref}

\label{name}

% References
\ref{name}

% Turn the hyperlink off:
% after \begin{document}
\makeatletter
\let\ref\@refstar
\makeatother


%==============================================================================
% Graphics
%==============================================================================
\usepackage{graphicx}


%==============================================================================
% Enumerate symbols
%==============================================================================
\usepackage{enumerate}

% 1. 2. etc.
\begin{enumerate}[1.]

% 1) 2) etc.
\begin{enumerate}[1)]

% OR

\usepackage{enumitem}

% 1) 2) etc.
\begin{enumerate}[label=\arabic*)]

% a) b) etc.
\begin{enumerate}[label=\alph*)]


%==============================================================================
% Numerate pages
%==============================================================================

% Format {\No of Chapter - \No of page}
\usepackage[auto]{chappg}



%==============================================================================
% Epigraphes
%==============================================================================

\usepackage{epigraph}
\epigraph{\textit{The greatest difficulties lie where we are not looking for them.}}
{-- Johann Wolfgang von Goethe}



%==============================================================================
% Table Content
%==============================================================================

\renewcommand\contentsname{Оглавление} %%% renaming the Table of Contents

%arabic|alpha|roman
\renewcommand{\thechapter}{\arabic{chapter}.} %Add point after Section
\renewcommand{\thesubsection}{\arabic{section}.\arabic{subsection}.}


%==============================================================================
% Tables
%==============================================================================

%\usepackage{longtable} %for multipage tables
%longtable instead of table
%h < h! < H
%pH - on new page
%th - on top of page
%bh - on bottom of page
\begin{table}[h]\caption{Name of table}\label{name}
	\begin{tabular}{|r|c{0.7\linewidth}|l|} % 70\% of page width
		1 & 2 3 \\

		% Union of two columns
		\multicolumn{2}{|c|}{Результаты измерений} & 3 \\
		\multirow{}{}
	\end{tabular}
\end{table}


% Line between rows
\hline

% Line in the rows
\cline{fromRow-tillRow}


%Additional vertical space
\\[1cm]


% Rotate on 90 degrees
\rotate{90}{smt}


%==============================================================================
% Tikz diagrams
%==============================================================================
\usepackage{tikz}
\usetikzlibrary{positioning,arrows}

% Example:
\newcommand{\ArrowL}{2,0}
\begin{tikzpicture} [thick, node distance=2cm, text height=1.5ex, text depth=.25ex, auto]
	\node (Ca) {\ce{Ca}};
	\node[right of=Ca] (CaO) {\ce{CaO}};
	\node[right of=CaO] (CaOH)  {\ce{Ca(OH)2}};
	\node[right of=CaOH]  (CaSO) {\ce{CaSO4}};

	\node[below of=Ca] (CaS) {\ce{CaS}};

	\path[->] (Ca) edge (CaO);
	\path[->] (CaO) edge (CaOH);
	\path[->] (CaOH) edge (CaSO);
	\path[->] (Ca) edge (CaS);
	\path[->] (CaO) edge (CaS);
	\path[->] (CaOH.240) edge (CaS.-1);
	\path[<-] (CaOH) edge (CaS);

\end{tikzpicture}


%==============================================================================
% Chemistry chemfig
%==============================================================================

\usepackage{chemfig}

%\chemfig{<atom1><bond type>[<angle>,<coeff>,<tikz code>]<atom2>}
% coeff - the coefficient for multiplying the bond length

%Angles are counterclockwise:
% 0 = 0
% 1 = 45
% 2 = 90
% 3 = 135
% 4 = 180
% 5 = 225
% 6 = 270
% 7 = 315

% Bonds
%\chemfig{A-B} - single
%\chemfig{A=B} - double
%\chemfig{A~B} - triple
%\chemfig{A>B} - black from left to right
%\chemfig{A<B} - black from right to left
%\chemfig{A>:B} - pipes
%\chemfig{A<:B} - pipes
%\chemfig{A>|B} - white
%\chemfig{A<|B} - white

\chemfig{A-[1]B-[7]C}

% Ring
\chemfig{A*6(-B-C-D-E-F-)}
\chemfig{*6(=-=-=-)}
\chemfig{**5(------)}

% Opened ring
\chemfig{A*5(-B=C-D)}

% Ion
%^{-}
%^{\ominus}


%==============================================================================
% Chemistry mhchem
%==============================================================================

\usepackage[version=4]{mhchem}
%\ce{something}

\ce{SO4^2-}
\ce{NO3-}
\ce{AgCl2-}
\ce{^{227}_{90}Th+}
\ce{[AgCl2]-}
\ce{RNO2^{-.}}
\ce{$\mu\hyphen$Cl}

%- - single bond
%= - double bond
%# - triple bond (\bond{#})
% or
\ce{A\sbond B\dbond C\tbond D}

% v => arrow down
% ^ => arrow up
% -> => arrow right
% <- => arrow left
% <=> => double side arrow
% <-> => double side arrow from one line
% ->[\alpha][\beta] => \a upon the arrow \b down the arrow
% ->T[text]
% ->C[chemical]

% * or . => \cdot

